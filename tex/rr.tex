\documentclass[usenatbib]{mn2e}
\usepackage{graphicx}
\usepackage{bm}
\usepackage{fixltx2e}
\usepackage{astrobib_mnras2e}
%\usepackage{lineno}

\begin{document}
\topmargin-1cm
%\linenumbers

%Make my life significantly easier
\newcommand{\bd}{{\bm \delta}}


\title[Geometric Integrals for Correlation Functions]
{On Calculating the Geometric Integrals in Correlation Function Estimators}
\author[Padmanabhan\&Parejko]{Nikhil Padmanabhan$^{1}$, John Parejko$^{1}$ or
reversed \\
$^{1}$ Dept. of Physics, Yale University, New Haven, CT 06511 \\
}

\date{\today}
\maketitle

\begin{abstract}
  We present a quasi-Monte Carlo technique for computing the geometric integrals - $RR$ and $DR$ - that appear
  in pair-counted correlation function estimators. We demonstrate that this technique can accelerate the convergence of
  these integrals, substantially reducing the number of random points required
  to reach a desired error target. MAYBE PUT IN EXACT SCALINGS, DEPENDING ON
  WHAT WE FIND??? We also present a simplification for surveys with separable
  angular and radial window functions. ????
\end{abstract}

\section{Introduction}

\section{The RR Integral}

\subsection{The Landy-Szalay Estimator and the Standard Calculation}

\subsection{Three Examples}

\subsubsection{The Unit Interval}

\subsubsection{The Unit 3D Hypercube}

\subsubsection{The DEEP2 Survey Mask}

\section{The Algorithm}

\section{Separable Window Functions}

\section{Conclusions}

\section{Acknowledgments}

\appendix

\section{A review of Monte-Carlo and Quasi-Monte-Carlo Integration}


\end{document}




