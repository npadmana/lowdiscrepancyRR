\documentclass[usenatbib]{mn2e}
\usepackage{graphicx}
\usepackage{bm}
\usepackage{fixltx2e}
\usepackage{astrobib_mnras2e}
%\usepackage{lineno}

\begin{document}
\topmargin-1cm
%\linenumbers

%Make my life significantly easier
\newcommand{\vx}{{\bm x}}
\newcommand{\vr}{{\bm r}}
\newcommand{\vdx}{{\bm dx}}
\newcommand{\vy}{{\bm y}}
\newcommand{\bin}{\Theta}


\title[Geometric Integrals for Correlation Functions]
{On Calculating the Geometric Integrals in Correlation Function Estimators}
\author[Parejko\&Padmanabhan]{John Parejko$^{1}$, Nikhil Padmanabhan$^{1}$ or
reversed \\
$^{1}$ Dept. of Physics, Yale University, New Haven, CT 06511 \\
}

\date{\today}
\maketitle

\begin{abstract}
  We present a quasi-Monte Carlo technique for computing the geometric integrals - $RR$ and $DR$ - that appear
  in pair-counted correlation function estimators. We demonstrate that this technique can accelerate the convergence of
  these integrals, substantially reducing the number of random points required
  to reach a desired error target. MAYBE PUT IN EXACT SCALINGS, DEPENDING ON
  WHAT WE FIND??? We also present a simplification for surveys with separable
  angular and radial window functions. ????
\end{abstract}

\section{Introduction}

The correlation function is one of the fundamental measurements in
characterizing the distribution of galaxies, both on small and large scales.
Given its importance, ????? DISCUSS FAST METHODS HERE ETC ETC ETC ???

BRIEF DESCRIPTION OF THIS WORK

This paper is organized as follows : Sec.~\ref{sec:RR} introduces the $RR$
integral, and presents the usual method by which it is estimated. We then
introduce our suggested technique for calculating this integral, constrasting it
with the usual calculation in a series of three toy problems. We then summarize
the algorithm in Sec.~\ref{sec:alg} and present a realistic examples from
the DEEP2 DR4 (CITE???) survey. We then present an
important simplification applicable to surveys with separable 
angular and radial selection functions in Sec.~\ref{sec:sep}.
We continue by extending our algorithm to the $DR$ integral in
Sec.~\ref{sec:DR}. Appendix~\ref{sec:review} reviews the basics of Monte Carlo and quasi-Monte
Carlo integration techniques, and gathers together results used in this work.

A note on the nomenclature used in the paper. Vectors are denoted as ($\vx$),
with their components specified as $x_{i}$. We also define a bin function
$\bin(\vx,\vy)$ which is $1$ if the separation between $\vx$ and $\vy$
corresponds to the bin of interest, and zero otherwise. 

\section{The RR Integral}
\label{sec:RR}

We start by reviewing the basics of galaxy correlation function estimation, as 
is most commonly used in large-scale structure work today. We then introduce 
our modifications in a series of toy examples, that highlight both the algorithm
and the differences with the traditional approach.

\subsection{Correlation Function Estimators and their Standard Calculation}

The galaxy two-point correlation $\xi(r)$ is defined as the fractional excess
over a uniform distribution in the number of galaxy pairs separated by a
distance $r$ from one another. The most commonly used estimator is the
Landy-Szalay (CITE???) estimator :
\begin{equation}
\xi = \frac{DD - 2 DR + RR}{RR}
\end{equation}
where $DD$ is the number of galaxy pairs. This raw number of pairs must be
scaled and corrected for the effects of the survey selection function. The $DR$
and $RR$ terms represent this correction. $RR$ is the expected number of pairs
if the galaxies were homogeously distributed according to the survey selectio
function, while $DR$ is the expected number of pairs around the actual galaxies
themselves. If we consider the survey selection function $n(\vx)$ normalized
by the volume integral
\begin{equation}
N = \int d\vx\,n(\vx)
\end{equation}
where $N$ is the observed number of galaxies, then 
\begin{equation}
RR = \int d\vx_1  \int d\vx_2 \,n(\vx_1) n(\vx_2) \bin(\vx,\vy) \,\,,
\label{eq:RRdef}
\end{equation}
and 
\begin{equation}
DR = \sum_{i=1}^{N} \int d\vx_1 \,n(\vx_1) \bin(\vx_1, \vx_{g,i}) \,\,
\label{eq:DRdef}
\end{equation}
where the sum for $DR$ runs over the $N$ galaxy positions $\vx_{g,i}$. 
Note that for an $n$-dimensional galaxy distribution, the $RR$ integral is
$2n$-dimensional, while $DR$ is $n$-dimensional. In what follows, it is often
convenient to scale $DR$ and $RR$ as defined by $N^2$; for simplicity, we refer
to these as $DR$ and $RR$ as well.

These integrals are usually evaluated by Monte-Carlo techniques by laying down a
set of random points according to the survey selection function and then
counting pairs as with the actual galaxies. 

\subsection{Three Toy Problems}

\subsubsection{The Unit Interval}

\subsubsection{A Spherical Cap}

\subsubsection{The Unit 3D Hypercube}

\section{The Algorithm Summarized}
\label{sec:alg}

PSEUDO-CODE HERE?????
\begin{enumerate}
  \item Generate a random vector in $[0,1)^{2n}$ .
  \begin{enumerate}
    \item Repeat the instructions in this group $N$ times.
    \item Generate a $2n$ dimensional element in $[0,1)^{2n}$ from a low
    discrepancy sequence.
    \item Shift it (mod 1) by the random vector stored above.
    \item Generate $\vx$ from the first $n$ elements and $\vdx$ from the last
    $n$.
    \item Generate $\vy$ from $\vx$ and $\vdx$. The most common rules are
    summarized below.
    \item Evaluate $n_1(\vx) n_2(\vy) \bin(\vx, \vy)$ and add to a running sum
    $\Sigma$.
  \end{enumerate}
  \item The estimate for $RR$ is then $\Sigma/N$ multiplied by the appropriate
  Jacobian factors from the variable transformations. 
  \item If desired, an error may be estimated by repeating this process and
  measuring the scatter. 
\end{enumerate}

COMMON VARIABLE TRANSFORMATIONS
\begin{itemize}
  \item $2D$ Cartesian space
  \item $2D$ Spherical coordinates
  \item $3D$ Cartesian space
\end{itemize}

\subsection{The DEEP2 DR4 Survey Mask}

\begin{figure}
\includegraphics[width=3in]{plots/deep2mask}
\caption{The DEEP2 survey mask for the first pointing in field 3. The
completeness ranges from 0 (white) to 1 (black). }
\label{fig:deep2mask}
\end{figure}



\section{Separable Window Functions}
\label{sec:sep}

\subsection{Application to the SDSS-III Survey Mask}

\section{The DR Integral}
\label{sec:DR}

\section{Conclusions}
\label{sec:conclude}


\section{Acknowledgments}

\appendix

\section{A review of Monte-Carlo and Quasi-Monte-Carlo Integration}
\label{sec:review}

\end{document}




