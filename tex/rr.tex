%% LyX 2.0.3 created this file.  For more info, see http://www.lyx.org/.
%% Do not edit unless you really know what you are doing.
\documentclass[usenatbib]{mn2e}
\usepackage[latin9]{inputenc}
\setcounter{tocdepth}{3}
\usepackage{graphicx}
\usepackage{esint}

\makeatletter
%%%%%%%%%%%%%%%%%%%%%%%%%%%%%% User specified LaTeX commands.
\usepackage{bm}
\usepackage{fixltx2e}
\usepackage{astrobib_mnras2e}
%\usepackage{lineno}

% Filler text
\usepackage{lipsum}

\topmargin-1cm
%\linenumbers

%Make my life significantly easier
\newcommand{\vx}{{\bm x}}
\newcommand{\vr}{{\bm r}}
\newcommand{\vdx}{{\bm dx}}
\newcommand{\vy}{{\bm y}}
\newcommand{\bin}{\Theta}

\makeatother

\begin{document}
\title[Geometric Integrals for Correlation Functions]
{On Calculating the Geometric Integrals in Correlation Function Estimators}

\author[Padmanabhan\&Parejko]{Nikhil Padmanabhan$^{1}$, John K. Parejko$^{1}$ \\
$^{1}$ Dept. of Physics, Yale University, New Haven, CT 06511 \\
}

\date{\today}

\maketitle
\begin{abstract}
We present a quasi-Monte Carlo technique for computing the geometric
integrals - $RR$ and $DR$ - that appear in pair-counted correlation
function estimators. We demonstrate that this technique can accelerate
the convergence of these integrals, substantially reducing the number
of random points required to reach a desired error target. MAYBE PUT
IN EXACT SCALINGS, DEPENDING ON WHAT WE FIND??? We also present a
simplification for surveys with separable angular and radial window
functions. ???? 
\end{abstract}

\section{Introduction}

The correlation function is one of the fundamental measurements in
characterizing the distribution of galaxies, both on small and large
scales. Given its importance and the computationally intensive nature
of the na\"ive $\mathcal{O}(N^{2})$ calculation, a number of techniques
have been developed to speed up this calculation \citep[e.g.][MORE!!!]{1980lssu.book.....P,2001misk.conf...71M}.
These fast estimation and counting methods are particularly necessary
for counting the RR pairs, as a large on-sky region may require, at
minimum, several million pseudo-random points to properly map the
geometry \citep[e.g.][]{2012MNRAS.427.3435A,2013MNRAS.429...98P},
resulting $>10^{12}$ potential pairs. Additionally, a pseudo-random
distribution of N points does not optimally map the geometry, resulting
in a higher variance than might be possible with a more favorable
distribution.

In this work, we describe a technique for calculating RR (and, by
extension, DR) using low-discrepency sequences. A low-discrepency--alternately,
``quasi-random''--sequence in some measure space has the fraction
of points falling into an arbitrary set nearly proportional to the
measure of that set. A particular sequence of uniformly distributed
pseudo-random numbers, as are typically used in the RR calculation,
does not have low discrepency, though the uniform distribution that
the pseudo-random points draw from does.

MORE BRIEF DESCRIPTION OF THIS WORK

This paper is organized as follows : Sec.~\ref{sec:RR} introduces
the $RR$ integral, and presents the usual method by which it is estimated.
We then introduce our suggested technique for calculating this integral,
constrasting it with the usual calculation in a series of three toy
problems. We then summarize the algorithm in Sec.~\ref{sec:alg}
and present a realistic examples from the DEEP2 DR4 (CITE???) survey.
We then present an important simplification applicable to surveys
with separable angular and radial selection functions in Sec.~\ref{sec:sep}.
We continue by extending our algorithm to the $DR$ integral in Sec.~\ref{sec:DR}.
Appendix~\ref{sec:review} reviews the basics of Monte Carlo and
quasi-Monte Carlo integration techniques, and gathers together results
used in this work.

A note on the nomenclature used in the paper. Vectors are denoted
as ($\vx$), with their components specified as $x_{i}$. We also
define a bin function $\bin(\vx,\vy)$ which is $1$ if the separation
between $\vx$ and $\vy$ corresponds to the bin of interest, and
zero otherwise.


\section{The RR Integral}

\label{sec:RR}

We start by reviewing the basics of galaxy correlation function estimation,
as is most commonly used in large-scale structure work today. We then
introduce our modifications in a series of toy examples, that highlight
both the algorithm and the differences with the traditional approach.


\subsection{Correlation Function Estimators and their Standard Calculation}

The galaxy two-point correlation $\xi(r)$ is defined as the fractional
excess over a uniform distribution in the number of galaxy pairs separated
by a distance $r$ from one another. The most commonly used estimator
is the Landy-Szalay \citep{1993ApJ...412...64L} estimator : 
\begin{equation}
\xi=\frac{DD-2DR+RR}{RR}
\end{equation}
where $\xi$ here represents the correlation function in a particular
separation bin (appropriately averaged) and $DD$ is the number of
galaxy pairs that fall in that bin. This raw number of pairs must
be scaled and corrected for the effects of the survey selection function.
The $DR$ and $RR$ terms represent this correction. $RR$ is the
expected number of pairs if the galaxies were homogeously distributed
according to the survey selection function, while $DR$ is the expected
number of pairs around the actual galaxies themselves. If we consider
the survey selection function $n(\vx)$ normalized by the volume integral
\begin{equation}
N=\int d\vx\, n(\vx)
\end{equation}
where $N$ is the observed number of galaxies, then 
\begin{equation}
RR=\int d\vx_{1}\int d\vx_{2}\, n(\vx_{1})n(\vx_{2})\bin(\vx,\vy)\,\,,\label{eq:RRdef}
\end{equation}
and 
\begin{equation}
DR=\sum_{i=1}^{N}\int d\vx_{1}\, n(\vx_{1})\bin(\vx_{1},\vx_{g,i})\,\,\label{eq:DRdef}
\end{equation}
where the sum for $DR$ runs over the $N$ galaxy positions $\vx_{g,i}$.
Note that for an $n$-dimensional galaxy distribution, the $RR$ integral
is $2n$-dimensional, while $DR$ is $n$-dimensional. In what follows,
it will be convenient to scale $DR$ and $RR$ as defined by $1/N^{2}$,
equivalent to normalizing the selection function by $\int d\vx\, n(\vx)=1$.
For simplicity, we refer to these as $DR$ and $RR$ as well.

In what follows, it is important to remember that both $DR$ and $RR$
are purely geometric quantities and do not have any intrinsic statistical
fluctuations. These therefore should be estimated at a high enough
accuracy to introduce no additional errors to the correlation function
estimate.


\subsection{Calculating RR}

Given the complexity of survey selection functions, these integrals
are most conveniently estimated by Monte-Carlo techniques. We continue
to adopt this approach here, and only depart in the details of our
implementation. We focus on the $RR$ calculation here, and discuss
the $DR$ calculation in Sec. \ref{sec:DR}.

The traditional approach to calculating $RR$ is to lay down a set
random points according to the survey selection function and then
to count the number of pairs when correlating this random catalog
with itself. This approach has much to recommend it: it is identical
to the $DD$ calculation and therefore conceptually simple to implement,
and the random catalogs so created are a convenient and portable representation
of the survey selection function. However, the number of random points
needs to be significantly larger than the number of data points to
prevent the estimation error in $RR$ from contributing to the error
in $\xi$; factors of 50 to 100 times the data sample are relatively
commonly used. A result of this is that the calculation of the correlation
function is immediately dominated by the $RR$ pair counting, and
significant effort has been expended to speed up this calculation,
both with algorithmic improvements \citep[e.g.][]{2001misk.conf...71M}
and parallelization (CITES????).

\begin{figure*}
\includegraphics[width=1.5in]{plots/grid1d-1} \includegraphics[width=1.5in]{plots/grid1d-2}
\includegraphics[width=1.5in]{plots/grid1d-3} \includegraphics[width=1.5in]{plots/grid1d-4}
\caption{Computing $RR$ for a 1D survey geometry on the unit interval for
points separated by 0.2. The area of the shaded region is the desired
value of $RR$. Starting from the left, the panels are (a) the traditional
algorithm, formed by the Cartesian product of a 1D random set, (b)
pseudo-random 2D numbers, (c) a 2D Niederreiter quasi-random sequence
and (d) a 2D Niederreiter sequence in position and displacement (see
text). Note that the $y$-axis in the last panel runs from -0.2 to
1.2 to account for points on the edges. The number of points (pairs)
in all panels is the same.}


\label{fig:grid1d} 
\end{figure*}


The inefficiencies with this approach is clearly demonstrated by a
toy example. Consider a 1D survey geometry, and imagine we needed
to estimate $RR$ for points separated by less than a chosen distance.
Fig.~\ref{fig:grid1d} graphically demonstrates this for a survey
defined on the unit interval, and a separation distance of 0.2. Calculating
$RR$ is equivalent to the problem of calculating the area of the
shaded region. The leftmost panel illustrates the algorithm as described
above. Since the pairs are the Cartesian product of the random set,
they arrange themselves onto a grid, and cover the 2D integral domain
very non-uniformly. As we will see explicitly below, this adversely
affects the convergence rate of this algorithm, scaling much slower
than the $\sqrt{N_{{\rm pairs}}}$ that one would expect for a random
sequence.

Our first improvement is shown in the second panel - we simply randomly%
\footnote{Technically, pseudo-randomly, since the random number generator is
completely deterministic given a starting seed. We use the GSL implementation
of the Mersenne Twister throughout this paper.%
} distribute the points over the full integration domain. Both panels
have the same number of points sampling the integration domain, and
the difference in coverage is apparent. This is \textit{not} equivalent
to generating two separate random sets and correlating them - we are
sampling points from the underlying $2n$ dimensional space (2D in
this case). We expect this to scale as $\sqrt{N_{{\rm pairs}}}$ (see
below). However, the distribution of points in this case is still
non-optimal: large gaps and overdensities still remain.

We next replace the pseudo-random 2D distribution with a 2D Niederreiter
quasi-random sequence containg the same number of points (third panel).
This sequence has low discrepency, and clearly fills the space more
uniformly than either of the pseudo-random sequences.


\subsection{A Spherical Cap and a Unit Cube}

\lipsum[1-3]

\begin{figure}
\includegraphics[width=3in]{plots/cap2d} \caption{???The scaling of the relative error in RR for a spherical cap as
a function of the number of pairs. The cap is centered about the north
pole, and extends down by $60^{\circ}$. The circles and squares correspond
to separations of points $<0.1$ and $10$ degrees respectively. The
long and short dashed lines are $\propto N_{{\rm pairs}}^{-1/2}$
and $N_{{\rm pairs}}^{-1}$ respectively. The error scaling is steeper
than the $N^{-1/2}$ expected from purely random numbers.}


\label{fig:cap2d} 
\end{figure}


\begin{figure}
\includegraphics[width=3in]{plots/unit3d} \caption{???The scaling of the relative error in RR for a unit cube as a function
of the number of pairs. The circles and squares correspond to radial
bins of $[0.015625,0.03125]$ and $[0.25,0.5]$ respectively. The
long and short dashed lines are $\propto N_{{\rm pairs}}^{-1/2}$
and $N_{{\rm pairs}}^{-1}$ respectively. The error scaling is steeper
than the $N^{-1/2}$ expected from purely random numbers.}


\label{fig:unit3d} 
\end{figure}



\section{Real World Examples}


\subsection{The Algorithm Summarized}

\label{sec:alg}

PSEUDO-CODE HERE????? 
\begin{enumerate}
\item Generate a random vector in $[0,1)^{2n}$ . 

\begin{enumerate}
\item Repeat the instructions in this group $N$ times. 
\item Generate a $2n$ dimensional element in $[0,1)^{2n}$ from a low discrepancy
sequence. 
\item Shift it (mod 1) by the random vector stored above. 
\item Generate $\vx$ from the first $n$ elements and $\vdx$ from the
last $n$. 
\item Generate $\vy$ from $\vx$ and $\vdx$. The most common rules are
summarized below. 
\item Evaluate $n_{1}(\vx)n_{2}(\vy)\bin(\vx,\vy)$ and add to a running
sum $\Sigma$. 
\end{enumerate}
\item The estimate for $RR$ is then $\Sigma/N$ multiplied by the appropriate
Jacobian factors from the variable transformations. 
\item If desired, an error may be estimated by repeating this process and
measuring the scatter. 
\end{enumerate}
COMMON VARIABLE TRANSFORMATIONS 
\begin{itemize}
\item $2D$ Cartesian space 
\item $2D$ Spherical coordinates 
\item $3D$ Cartesian space 
\end{itemize}
\lipsum[1-3]


\subsection{Two Dimensions : The DEEP2 DR4 Survey Mask}

\begin{figure}
\includegraphics[width=3in]{plots/deep2mask} \caption{The DEEP2 survey mask for the first pointing in field 3. The completeness
ranges from 0 (white) to 1 (black). }


\label{fig:deep2mask} 
\end{figure}


\begin{figure}
\includegraphics[width=3in]{plots/deep2rrcomp1} \caption{???The scaling of the relative error in RR for the DEEP2 survey mask
as a function of the number of pairs. The circles and squares correspond
to the angular bins $[28.12,56.25]$ arcseconds and $[7.5,15]$ arcminutes
respectively, while the filled and open symbols compare a low discrepancy
sequence to pseudo-random numbers. The long and short dashed lines
are $\propto N_{{\rm pairs}}^{-1/2}$ and $N_{{\rm pairs}}^{-1}$
respectively. The low discrepancy sequences outperform pseudo-random
numbers both in the absolute error as well as in the scaling with
$N_{{\rm pairs}}$.}


\label{fig:deep2comp1} 
\end{figure}


\begin{figure}
\includegraphics[width=3in]{plots/deep2rrcomp2} \caption{The same as Fig.~\ref{fig:deep2comp1} except the open symbols are
now calculated using the traditional RR method. The dot-dashed line
is $\propto N_{{\rm pairs}}^{-1/4}$. The significant inefficiencies
in the traditional approach are clearly apparent here. }


\label{fig:deep2comp2} 
\end{figure}


\label{fig:deep2PDF}
\begin{figure}
\includegraphics[width=3in]{plots/deep2rrhist} \caption{????}


\label{fig:deep2comp2} 
\end{figure}


\lipsum[1-3]


\subsection{Three Dimensions : The BOSS LOWZ sample}

\lipsum[1-3]


\subsubsection{Geometry}

\lipsum[1-3]


\subsection{Separable Window Functions}

\label{sec:sep}

\lipsum[1-3]


\section{The DR Integral}

\label{sec:DR}

\lipsum[1-3]


\section{Conclusions}

\label{sec:conclude}

\lipsum[1-3]


\section{Acknowledgments}

\bibliographystyle{mn2e}
\bibliography{bib/clustering_method,bib/sdss-tech,bib/sdss,bib/Npoint,bib/lowdiscr}

\end{document}
